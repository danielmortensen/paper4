\section{Charger Management}
\par Limited numbers of chargers is another limitation that many transit authorities face.  Let the number of chargers be denoted $n_{\text{charger}}$. We desire to maintain the average cumulative power for each time step at a level that is serviceable given $n_{\text{charger}}$. We define a slack variable $p_c(j)$ which represents the total average power consumed by all buses at time $j$.  The variable $p_c(j)$ is computed as the sum of average bus powers so that
\begin{equation*}
	p_c(j) = \sum_ib_{p(i,j)}
\end{equation*}
or also as 
\begin{equation}
p_c(j) - \sum_ib_{p(i,j)}  = 0\ \forall j.
\end{equation}
\par From a practical standpoint, we must also avoid multiple charging sessions in one visit to the station, or charger thrashing. We to this by minimising the difference in the average power values throughout the day.  When a bus uses the charger for longer periods of time, the average power remains the same for multiple time periods.  If a bus reconnects/disconnects multiple times, there will be larger difference in the power use from one time period to another. This can be expressed as
\begin{equation}
	J_{\text{thrash}} = \sum_{i,j > 1} \lvert b_{p(i,j)} - b_{p(i,j-1)}\rvert.
\end{equation}
To minimize $J_{\text{thrash}}$, we define a slack variable $g_{i,j}$ so that 
\begin{equation*}
g_{i,j} = \lvert b_{p(i,j)} - b_{p(i,j-1)} \rvert.
\end{equation*}
which can be enforced by the linear constraints 
\begin{equation*}\begin{aligned}
	-g_{i,j} &\le b_{p(i,j)} - b_{p(i,j-1)} \\
	g_{i,j} &\ge b_{p(i,j)} - b_{p(i,j-1)}.
\end{aligned}\end{equation*}
Suppose $b_{p(i,j)} - b_{p(i,j-1)} = -5$, then we have 
\begin{equation*}
	-g \le -5 \le g
\end{equation*}
which minimizes $g$ when $g = 5$. A similar expression is minimized when $g$ is positive. The expression for $g_{i,j}$ can alternatively be expressed in standard form as
\begin{equation}\begin{aligned}
-g_{i,j} - b_{p(i,j)} + b_{p(i,j-1)} \le 0 \\
-g_{i,j} + b_{p(i,j)} - b_{p(i,j-1)} \le 0.
\end{aligned}\end{equation}

