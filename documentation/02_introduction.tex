\section{Introduction}
\par  Many transit authorities desire to use battery electric buses (BEBs) for public transportation because BEBs offer many benefits \cite{Mahmoud2016} including reduced maintenance \cite{poornesh_comparative_2020}, zero emissions \cite{kato_comparative_2013}, and access to renewable energy \cite{cheng_smart_2020}.
\par Unfortunately, BEBS cannot replace traditional buses without managing lengthy charge times. Historically, a bus using diesal or comparessed natural gas (CNG) refuels quickly so that fuel times play a minor role in their use. However charge times for BEBs can range up to several hours, which presents logistical challenges for bus fleets and necessitate detailed charge plans which address four points of interest: battery state, bus availability, charging resources, and the cost of using electrical infrastructure. 
\par Addressing battery state includes charging buses so that their state of charge remains above a minimum threshold. Futhermore, if computations are only done for one day, the state of charge for each bus at the end of the day must be the same as the beginning. Constraints for bus availability only allow charging when buses are near charging resources. 
\par For example, if all chargers are located at the bus station, then each bus's availability would mirror its time in the station. Unfortunately, proximity to charging resources alone does not gaurentee access because other buses may already occupy the chargers. 
\par Finally, when a bus does gain charger access, power providers assess charges for both energy and power so that high charge rates quickly become expensive. High charge rates are not limited to single buses either as the power and energy for each bus is combined at a single meter including loads that are not related to bus charging. For example, if two buses charged at 200 kW each, and an additional non-bus load drew 100 kW, then the power provider would assess a power charge based on 500 kW of use. In this work, we refer to the problem of forming a charge plan under the aforementioned constraints as the ``charge problem''.
\section{Previous Work}
\par The charge problem has received attention in previous work and is generally addressed in one of three ways: dynamic charging, battery swaps, and static charging.
\subsection{Dynamic charging}
\par Dynamic charging allows buses to utilize charging resources while in motion, generally through overhead charging lines \cite{csonka_optimization_2021}, or inductive power transfer \cite{jeong_automatic_2018} \cite{balde_electric_2019}. 
\par Dynamic charging increases access to charging resources because buses can refuel while in motion so that availability does not depend on time spent in the bus station. Unfortunately, both overhead and inductive chargers require additional hardare; overhead chargers through the overhead power lines, and inductive charging through specialized hardware beneath the road. Both of which require additional infrastructure \cite{Alwesabi_2022_Robust} which may be unavailable, or cost prohibitive.
\subsection{Battery Swaps}
\par Battery swaps manage the energy needs for each bus by replacing spent batteries so that they can be charged without delaying the buses as proposed by \cite{jain_battery_2020} and \cite{xian_zhang_optimal_2016}. Battery exchanges greatly simplify charge plans and relieve the need for dynamic infrastructure. The only drawback to battery swaps comes from how BEBs are constructed as BEBs are not built to exchange batteries. Therefore, the task can require specialized hardware, technical expertise, or automation.
\subsection{Static Charging}
\par A static charging scenario takes place when charging resources remain anchored so that buses must stop in order to refuel and is the least invasive way to manage a bus fleet's energy needs. Prior work in this area addresses a number of problems, including distributed charging networks \cite{Nimalsiri2020}, bus availability, environmental impact \cite{zhou_bi-objective_2021}, route scheduling \cite{Rinalde_Mixed_2020}, battery health \cite{houbbadi_optimal_2019}, the cost of electricity \cite{Leou_optimal_2017}, and the cost of charging infrastructure \cite{Wei2018}.
\par Because charge times can be lengthy, some prefer to use high power chargers, which deliver more energy in a smaller period of time. However doing so places large power demands on electrical infrastructure \cite{stahleder_impact_2019} so that power networks becomes unreliable \cite{deb_impact_2017} and expensive because high power requires additional maintenance and upgrades \cite{boonraksa_impact_2019}. An effective charge plan must therefore balance the need to charge quickly with the desire to maintain a low power profile \cite{ojer_development_2020}.
\par Methods for developing a charge plan range from heuristic approaches \cite{qin_numerical_2016} \cite{Wang2019} to globally optimal solutions, generally through mixed integer linear programs (MILP) \cite{bagherinezhad_spatio-temporal_2020}, where the first prioritizes computational time and the second focuses on cost. Generally, each method minimizes cost by either decreasing the instantaneous power needs for the fleet, or optimizing over time of use tarrifs \cite{He_2019_Fast}.  
\subsection{Contributions}
\par  This work extends previous methods with two primary contributions: Uncontrolled loads, and scalability for both computation time and cost. Uncontrolled loads are defined as non-BEB loads which impact monthly cost. For example, the Utah Transit Authority in Salt Lake City manages several loads including an electric train, a station for compressed natural gas (CNG), and electric buses. 
\par When the train arrives at the station, the regenerative brakes input power to the grid, which would complement high charge rates on a BEB charger. However the train requires significant power to accelerate and makes additional charging unwise when it departs\par Generally, previous work falls into one of two categories. The first prioritises computation time at the expense of cost, generally through heuristics. The second finds a globally optimal solution, but does so at the cost of computation time. This paper presents a method which accounts for both so that computation time and monthly cost increase linearly with the number of buses. 
