\section{Multi-Rate Charging}
Up to this point, we have computed the ``optimal'' schedule which assumes any bus can charge without regard to the number of chargers. We then separate buses into groups to reduce the scope of the problem and treat each sub-problem separately while we defragment and assign each charge session to specific chargers before determining the final start and stop times for each bus's charge session so that each bus maintains charge sessions at various times and chargers with specific energy requirements. 
\par The final step in this process is to determine how the energy will be delivered so that cost is minimised. Begin with constraints for bus power, energy, and cost from Section \ref{sec:optimalSolution} that are expressed as equations \ref{eqn:obj:power2}, \ref{eqn:battery:busPower}, \ref{eqn:objective:pt}, \ref{eqn:objective:p15}, \ref{eqn:objective:pFac}, \ref{eqn:objective:pDem} and \ref{eqn:objective:energy}. Next, include constraints for energy so that the energy for each charge session is properly delivered (see equation \ref{eqn:defragmentation:active}). Once this program is successfully run, it must be run again using the smoothing objective from Eqn. \ref{eqn:objective:smooth}.
