\subsection{Objective\label{sec:objective}}
\par Now that the relevant constraints have been addressed, we must work towards computing the total objective function. We do so by first computing the total average power for the complete system. This total power is comprised of power used by the buses, and power used by external sources such as lights, ice melt, electric trains, etc which we refer to as ``uncontrolled loads'', where the average power for the uncontrolled loads at time step $j$ is denoted $u(j)$. We compute the total power as the sum of power used by the buses, $p_c(j)$ and the power consumed by uncontrolled loads $u(j)$ so that the total power, denoted $p_t(j)$ is computed as 
\begin{equation}\label{eqn:objective:pt}
	p_t(j) = p_c(j) + u(j).
\end{equation}
\par The next step is to compute the fifteen minute average power use for each time step, denoted $p_{\text{15}}$. We do this by letting 
\begin{equation}\label{eqn:objective:p15}
p_{\text{15}}(j) = \frac{1}{n}\sum_{l \in \{j_{15}\}}p_t(l),
\end{equation}
where $\{j_{15}\}$ is the set of all indices 15 minutes prior to $j$ and $n$ is the cardinality of $\{j_{15}\}$.
Next, note that the rate schedule requires both the maximum overall average power, denoted $p_{\text{facilities}}$, and the maximum average power during on-peak hours, or $p_{\text{demand}}$. Let $\mathcal{S}_{\text{on}}$ be the set of time indices belonging to on-peak hours, and recall that the max over all average power values is greater than or equal to $p_{15}(j)$ for all $j$. We can express this constraint is
\begin{equation}\label{eqn:objective:pFac}
	p_{\text{facilities}} \ge p_{15}(j) \ \forall j.
\end{equation}
Because $p_{\text{facilities}}$ will be used in the objective function, the value for $p_{\text{facilities}}$ will be minimized until it is equal to the largest value in $p_{15}$. Following a similar logic, we also define a set of constraints for the maximum average on-peak power, $p_{\text{demand}}$ so that
\begin{equation}\label{eqn:objective:pDem}
	p_{15}(i) - p_{\text{demand}} \le 0 \ \forall i \in \mathcal{S}_{\text{on}}.
\end{equation}
The next step in computing the objective function is to compute the total {\it energy} consumed during on and off-peak hours respectively.  Let $e_{\text{on}}$ be the total energy consumed during on-peak hours and $e_{\text{off}}$ be the energy consumed during off-peak hours. We can compute energy as the product of average power and time.  In our case, we compute this as 
\begin{equation}\label{eqn:objective:energy}\begin{aligned}
	e_{\text{on}} &= \Delta T\cdot \sum_{i \in \mathcal{S}_{\text{on}}}p_t(i) \\ 
	e_{\text{off}} &= \Delta T\cdot \sum_{i \notin \mathcal{S}_{\text{on}}}p_t(i).  
\end{aligned}\end{equation}
We can now compute the total monthly cost in dollars as
\begin{equation}\label{sec:unconstrainedSchedule:objective}
J_{\text{cost}} = \begin{bmatrix}e_{\text{on}} \\ e_{\text{off}} \\ p_{\text{facilities}} \\ p_{\text{demand}} \end{bmatrix}^T \begin{bmatrix} \mu_{\text{e-on}} \\ \mu_{\text{e-off}} \\ \mu_{\text{p-all}} \\ \mu_{\text{p-on}} \end{bmatrix}, 
\end{equation}
where $\mu_{\text{e-on}}$, $\mu_{\text{e-off}}$, $\mu_{\text{p-all}}$, and $\mu_{\text{p-on}}$ represent the cost for on-peak energy, the cost of off-peak energy, the facilities rate, and demand charge respectively.
\par In summary, the problem $p_1$ described in Section \ref{sec:unconstrainedSchedule} computes a charge schedule without constraints on the number of chargers. We have also observed that the resulting charge commands tend to be either $0$ or $p_{\text{max}}$ which is difficult to implement and imparts additional stress on charging hardware. Before additional steps can be taken, a smoothed version of the solution for $p_1$ must be computed.
\\[0.1in] \begin{tikzpicture}
	\node[rectangle, rounded corners, fill=gray!8, draw=gray!60, minimum width=\columnwidth, minimum height=0.7in] at (0,0)(box){};
	\node at (0,0.2in)(title){\underline{Summary for $p_1$}};
	\node at ($(box.south)!0.6!(title.south)$)(text){$ 
	\underset{\mathbf{y}}{\text{Min}}  \ \eqref{sec:unconstrainedSchedule:objective} \  \text{subject to} \ \eqref{eqn:obj:power2} \ \text{--} \ \eqref{eqn:objective:energy}.% \ \eqref{eqn:battery:socPropagation}, \ \eqref{eqn:battery:soc}, \ \eqref{eqn:battery:busPower}, \ \eqref{eqn:cumulative:power}, \ \eqref{eqn:objective:pt}, \ \eqref{eqn:objective:p15}, \ \eqref{eqn:objective:pFac}, \ \eqref{eqn:objective:pDem}, \ \eqref{eqn:objective:energy}
$};
\end{tikzpicture} 
where $\mathbf{y}$ represents the variables of optimization for $p_1$.


\section{$p_2$: Unconstrained Smooth Schedule \label{sec:unconstrainedSmoothSchedule}} 
\par This section implements a smoothing criteria so that the ``on-off'' patterns from the first solution are softened. This is done by first solving the un-constrained charge problem as given. Next, the same problem is solved again but with two primary differences. The first is that the demand, facilities, on-peak energy, and off-peak energy are constrained so that they are equal to the values obtained in $p_1$ so that
\begin{equation}\label{eqn:unconstrainedSmooth:equivalence}\begin{aligned}
	e_{\text{on}} &= \tilde{e}_{\text{on}} \\
	e_{\text{off}} &= \tilde{e}_{\text{off}} \\
	p_{\text{facilities}} &= \tilde{p}_{\text{facilities}} \\
	p_{\text{demand}} &= \tilde{p}_{\text{demand}}.
\end{aligned}\end{equation}
Next, we define an alternative objective which incentivizes ``smooth'' transitions between time steps. 
\par This objective is defined as
\begin{equation}\label{eqn:objective:smooth}
	J_{\text{thrash}} = \frac{1}{n}\sum_{i,j, \in \mathcal{K}}\lVert b(i,j) - b(i,j-1) \rVert^2_2,
\end{equation}
where $\mathcal{K}$ is the set of all $i,j$ where bus $i$ may charge during time $j$ and $j - 1$.
\par The smoothed schedule computed in $p_2$ minimizes the cost of charging in a way that maintains smooth charge schedules but is undesirable because the charge sessions tend to be small so that many are required. Additionally, the current schedule does not account for the number of chargers and contention. Unfortunately, resolving these issues requires the use of binary variables and becomes intractable for large numbers of buses. Before the fragmentation and charger assignment problems can be addressed, we must first segment the buses into groups so that they can be processed separately which will better manage the computations for binary-centric problems.
\\[0.1in] \begin{tikzpicture}
	\node[rectangle, rounded corners, fill=gray!8, draw=gray!60, minimum width=\columnwidth, minimum height=0.7in] at (0,0)(box){};
	\node at (0,0.2in)(title){\underline{Summary for $p_2$}};
	\node at ($(box.south)!0.6!(title.south)$)(text){$ 
	\underset{\mathbf{y}}{\text{Min}} \ \eqref{eqn:objective:smooth} \ \text{Subject to} \ \eqref{eqn:obj:power2} \ \text{--} \ \eqref{eqn:objective:energy}, \ \eqref{eqn:unconstrainedSmooth:equivalence}
$};
\end{tikzpicture} 

